\documentclass[10pt,a4paper,titlepage]{report}
\usepackage[utf8]{inputenc}
\usepackage[french]{babel}
\usepackage[T1]{fontenc}
\usepackage{amsmath}
\usepackage{amsfonts}
\usepackage{amssymb}
\usepackage[left=2cm,right=2cm,top=2cm,bottom=2cm]{geometry}

%Pour supprimer le retrait en début de paragraphe et ajouter un espace vertical entre les paragraphes
\usepackage{parskip}

%Pour police sans empattement des titres et chapitres
\usepackage[sf]{titlesec}

\author{Jean-Michel CHEREL}
\title{Descriptif des fonctions}

\begin{document}

%Pour police sans empattement
\sffamily

\chapter{Données de l'étude}

\section{Noeud(s)}

\section{Poutre(s)}

\section{Section(s) droite(s)}

\section{Matériau(x)}

\section{Liaison(s) nodale(s)}

\section{Chargement(s)}

\subsection{Cas de charge n°1}

\section{Combinaison(s)}

\subsection{Combinaison n°1}

\chapter{Résultats des calculs}

\chapter{Vérification des barres}

\section{Barre B1}

Norme de référence : NF EN 1993-1-1:2005/NA:2013/A1:2014, Eurocode 3

Point : 4

Position : 0.4*l = 1.116

\subsection{Chargements}

Cas de charge décisif : ELU02

\subsection{Matériau}

Acier S235

\subsection{Paramètre de la section}

T60X7

$S = 7.94$

$I{g,z} = 12.2$

$H = 6.0$

$B = 6.0$

$V{z} = 4.343$

$I{g,y} = 23.8$

$V{y} = 23.8$

\subsection{Efforts internes}

$N{ed} = -52.7 N$

$V{ed} = -154.58 N$

$M{ed} = 618.13 N.m$


\subsection{Paramètres de déversement}

Vérification du risque de déversement selon la méthode simplifiée pour les poutres avec maintiens latéraux dans les bâtiments, définie au 6.3.2.4 de la NF EN 1993-1-:2005

$\lambda {c0}$ = (Élancement limite de semelle comprimée équivalente, cf. clause 6.3.2.4(1) B de l'annexe nationale à la NF EN 1993-1-1:2005)


$I{f,z} = 1.61 cm$ (Rayon de giration par rapport à l'axe faible de la semelle comprimée équivalente composée de la semelle comprimée plus 1/3 de la partie comprimée de l'âme)

Sans objet

\subsection{Paramètres de flambement}

Sans objet

\subsection{Formules de vérification}

Contrôle de la résistance de la section :

Contrôle de la stabilité de la barre :

\subsection{Déplacements}

Sans objet

\subsection{Résultat}

RESULTAT

VERIFBARRES

\end{document}